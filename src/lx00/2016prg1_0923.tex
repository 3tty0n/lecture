\documentclass[a4paper]{article}
\usepackage{xeCJK}
\usepackage{url}
\usepackage{hyperref}
\defaultfontfeatures{Ligatures=TeX}

\setCJKmainfont{IPAMincho}
\setCJKsansfont{IPAGothic}
\setCJKmonofont{IPAGothic}


%\setmainfont{}
\setsansfont{URW Gothic}
\setmonofont{Inconsolata}

\usepackage{listings}

\title{計算機科学第一}
\date{9月23日}

\begin{document}
\maketitle

\noindent
今日の目標
\begin{itemize}
\item GitHub アカウントを作る。
\item 自分の情報をメールで送る。
%\item バージョン管理システム Git の \texttt{clone}, \texttt{push}, \texttt{pull} 及びその他のコマンドが使えるようになる。
\end{itemize}

\if0
%\section{今日の課題}
\noindent
今日の課題
\begin{enumerate}
%\item SSH鍵を GitHub アカウントに登録する。
%\item メールで送られている Invitation を受け入れて titech-is-cs115 の students グループのメンバーになる。
%\item GitHub 上にある \verb|titech-is-cs115/students| リポジトリの \verb|members/| ディレクトリの下に自分の名前のファイルを作成し自己紹介等を書く。
%そのために git コマンド clone, add, commit, push 等を用いよ。
%を clone する。
%\begin{verbatim}
%git clone https://github.com/titech-is-cs115/students.git
%\end{verbatim}
%\item リポジトリ students の中の members ディレクトリの下に自分の名前のファイルを作成する。
%ファイルの内容は氏名と自己紹介など。
%\item GitHub 上にある titech-is-cs115/students リポジトリの wiki にウェブブラウザでアクセスして課題のURLをクリックする。
%すると titech-is-cs115 の下に新しいリポジトリが作成される。
%\item そのリポジトリを clone して課題をやる。
%\begin{verbatim}
%git clone https://github.com/titech-is-cs115/assignment0-.git
%\end{verbatim}
%\item GitHub で配布されている今回の課題(sbt と ScalaTest を用いた単体テスト)をやる。提出締切は今週金曜日 23時59分59秒。
\end{enumerate}
\fi

\noindent
今日のワークフロー
\begin{enumerate}
\item GitHub アカウントを作る。
\item SSH鍵を作る。
\item SSH鍵を GitHub アカウントに登録する。
\item GitHub の \texttt{is-prg1a/lecture} にアクセスして scala ファイルを手に入れる。
\item Scalaファイルを適切に書き換えてから実行し、表示された指示に従ってメールを送る。
%\item メールで送られた Invitation を受け入れて titech-is-cs115 の students グループのメンバーになる(既にメンバーになっている人はスキップ)。
%\item GitHub 上にある \verb|titech-is-cs115/students| リポジトリの \verb|members/| ディレクトリの下に自分の名前のファイルを作成し自己紹介等を書く。
%そのために Git コマンド clone, add, commit, push 等を用いる。
%を clone する。
%\begin{verbatim}
%git clone https://github.com/titech-is-cs115/students.git
%\end{verbatim}
%\item リポジトリ students の中の members ディレクトリの下に自分の名前のファイルを作成する。
%ファイルの内容は氏名と自己紹介など。
%\item GitHub 上にある \verb|titech-is-cs115/students| リポジトリの wiki にウェブブラウザでアクセスして課題のURLをクリックする。
%すると titech-is-cs115 の下に\verb|assignment0-#アカウント名#|という名前の新しいリポジトリが作成される。
%\item そのリポジトリを clone して単体テストの課題をやる。最後に忘れず push する。
%\begin{verbatim}
%git clone https://github.com/titech-is-cs115/assignment0-.git
%\end{verbatim}
%\item GitHub で配布されている今回の課題(sbt test)をやる。
%\item GitHub 上にある \verb|titech-is-cs115/assignment0| リポジトリを自分のアカウントに fork する。
%\item Fork によって作成された \verb|#自分のアカウント名#/assignment0| リポジトリをローカルに clone する。
%この時にできたローカルリポジトリでまず sbt を起動しておいてから単体テストの課題に取りかかる(sbt の起動に時間がかかるので)。
%\item 単体テストの課題を終わらせて\verb|#自分のアカウント名#/assignment0| リポジトリに push する。
%\item Fork によって作成されたリポジトリ \verb|#自分のアカウント名#/assignment0| から
%Fork元のリポジトリ \verb|titech-is-cs115/assignment0| に Pull request を送る。
\end{enumerate}

\section{SSH 鍵の GitHub への登録}
この授業の課題の配布、提出は GitHubを通じて行います。
Git から GitHub 上のリポジトリにアクセスする度にパスワードを入力するのは面倒なので SSH 鍵を登録しておきます。
ここで SSHとは Secure Shell の略で暗号化された通信を実現するための仕組みです。
まず SSH 鍵を生成するために
\begin{verbatim}
ssh-keygen
\end{verbatim}
とコマンドを実行します。
すると
\begin{verbatim}
Generating public/private rsa key pair.
Enter file in which to save the key (/home/mori/.ssh/id_rsa): 
\end{verbatim}
と鍵ファイルのパスを尋ねられますがデフォルトのままでよいので、そのまま Enter を押します。
次に
\begin{verbatim}
Enter passphrase (empty for no passphrase): 
Enter same passphrase again: 
\end{verbatim}
とパスフレーズ(鍵を使用するときのパスワード)を尋ねられますが、これは空にしたいので何も入力せずに Enter を押します。
そうすると
\begin{verbatim}
Your identification has been saved in /home/mori/.ssh/id_rsa.
Your public key has been saved in /home/mori/.ssh/id_rsa.pub.
The key fingerprint is:
3c:08:1f:10:a4:c1:4e:46:cf:5a:6d:ca:8c:58:ad:2b mori@gmac01.is.titech.ac.jp
The key's randomart image is:
+--[ RSA 2048]----+
| oo.+.           |
|  +* o           |
| +o * +          |
| o.B = +         |
|. + + o S        |
|   .     .       |
|E .              |
| .               |
|                 |
+-----------------+
\end{verbatim}
というように表示されて鍵が生成されます。
ここで \texttt{id\_rsa} が秘密鍵、\texttt{id\_rsa.pub}が公開鍵です(両方ともテキストファイルです)。
公開鍵は暗号化に用いられ秘密鍵は復号に用いられます。秘密鍵は他人に知られてはいけません。公開鍵は暗号化された通信を
するために通信の相手に教える必要があります。公開鍵は第三者に知られても安全です。
この授業でこれ以上公開鍵暗号の説明はしませんが、「秘密鍵は信用できない者に知られてはいけない」、「公開鍵は誰に知られたとしても安全」
ということは知っておいてください。

次に公開鍵を GitHub に登録します。
GitHub にログインして右上にあるアイコンから Settings を選びます。
左のメニューから SSH keys を選びます。
そこで Add SSH key を選び、Title  とKey (公開鍵)を入力します。
Title はなんでもよいです(ひょっとしたら空でもよいかも)。公開鍵として \texttt{id\_rsa.pub} の内容を入力します。
ホームディレクトリで
\begin{verbatim}
cat .ssh/id_rsa.pub
\end{verbatim}
とコマンドを実行すると
\begin{verbatim}
ssh-rsa AAAAB3NzaC1yc2EAAAADAQABAAABAQC/gxywsteOQMka+SQRuSboMkamcKTp16
s1Kaac6GsdSIhZeJNfn+j/Ei9HOR7kg94ENIon2FHhgAffMtMIno9HZiiE+32ynxBf5trL
AgvzBnzDTu8PMfz3uxH6Yai5MDVufBlT++A42fwhxQQGcF4rmO77/2LWXSwTijGk7W2ji8
0OdBvZgmhwQNNg5LPa8x9JsPt4E3LgZPEREaVyxmPJzJFohRXLvMyqBtft3F60Qb7hAnrP
mUssRGLqxt8ah39HL/nN2t1KXMx3UH2pLMgxvxv/K6hhItX93vQM88YLtL8E+dB14Tp7lk
DxshNfgaA+qhlWrFgd98OLsvCeEMtb mori@gmac01.is.titech.ac.jp
\end{verbatim}
というようにファイルの中身が表示されます。
これをコピーアンドペーストで Key のところに入力してください。
%
これで GitHub に SSH を通じてアクセスできるようになりました。
もしも自宅など別の環境から GitHub にアクセスしたい場合には別途 SSH鍵を作成して登録してください(GitHubには複数の公開鍵を登録できます)。

\textbf{[この段落は余談です]} SSH はもともとリモートマシンに暗号化された通信用いて安全にログインするための仕組みです。
SSH を使えば西7号館の演習室に外からログインすることも可能です。
設定は演習室で \texttt{\~{}/.ssh/authorized\_keys} というファイルを作成して自宅で使用するSSH鍵の公開鍵を書き込むだけです(\texttt{\~{}/}はホームディレクトリという意味)。
これで自宅など他の場所から
\begin{verbatim}
ssh #アカウント名#@porto.is.titech.ac.jp
\end{verbatim}
で演習室のマシンにログインできるようになります(Windows の場合は何か SSH クライアントをインストールして使う必要があります)。
この場合に使用する秘密鍵は絶対に他人に知られないようにしてください。
秘密鍵を知られてしまうと演習室のマシンにログインされてしまいます。


\section{履修者の情報の登録}
\begin{enumerate}
\item GitHub アカウントを作ったら \url{https://github.com/is-prg1a/lecture} にアクセスしてください。
\item そうしてから \texttt{src/lx00/register.scala} をエディタにコピペして、内容を適切に書き換えて保存してください。
\item Scalaのプログラムファイルを \texttt{scala register.scala} で実行して、表示された内容に従ってメールを送信してください。
\end{enumerate}



\end{document}
